%\begin{landscape}
%\onecolumn
\scalefont{0.7}
\begin{longtable}{|p{2.3cm}|p{14cm}|}
\caption{Definitions of interoperability types}
\label{tab:understanding}
\hline
\textbf{Interoperability type} & \textbf{Definition} \\
\hline\hline
\endfirsthead
\multicolumn{2}{c}%
{\tablename\ \thetable\ -- \textit{Continued from previous page}} \\
\hline
\textbf{Interoperability type} & \textbf{Understanding} \\
\hline\hline
\endhead
\hline \multicolumn{2}{r}{\textit{Continued on next page}} \\
\endfoot
\hline
\endlastfoot
    Business
    &
    
    It works harmonized to share and develop business between companies despite the difference in methods, decision making, and the culture of enterprises [S2].
    
    It involves working harmoniously at the company and organizational levels despite different modes of decision making, work practices, culture,legislations, commercial approaches, and so on [S4].
    
    It is related to the strategic and organizational levels. This correlates to  BIM (Building Information Modelling)  because the use of  BIM is usually a  strategic action in the company.  Stakeholders need to be involved in the adoption process [S19].
    \\\hline 
    Business Process
    &
    It is associated with the functional aspects, such as workflow, that must be defined to share healthcare data between different countries effectively. It contributes to solving the current challenging issue—the lack of organizational interoperability [S27].
\\\hline 
    Cloud  & It defines the ability of cloud services to be able to work together with both different cloud services and providers, and other applications or platforms that are not cloud-dependent [S4], [S6].
    \\\hline
    Coalition
    &
    About technical and organizational aspects of interoperability. In this way, overcoming the technical interoperability gaps helps to identify organizational means that can increase the coalition’s interoperability when there is a lack of technical interoperability [S8].
    
    It is definitely not limited to the technical domain but also depends on organizational  [S21]. Coalition should deals with political, aligned procedures, and operations, and harmonized strategies /doctrines (paper Roiginal do LCI) 

    

    Aims to del
\\\hline    
    Conceptual
    &
     At this level, the systems are completely aware of each other's information, processes, contexts, and modeling assumptions [S17].
\\\hline
    Constructive 
    & 
    It is the ability of organizations responsible for constructing or maintaining a system to cooperate [S14].
    
    It addresses those activities related to the construction and maintenance of one system in the context of another system [S21].
\\\hline
    Cultural &  
    It is the degree to which knowledge and information are anchored to a unified model of meaning across cultures. Enterprise systems that take into consideration cultural interoperability aspects can be used by transnational groups in different languages and cultures with the same domain of interest in a cost-effective and efficient manner [S4] [S6].
\\\hline
     Data
     &
     It works with different data models and query languages to share information from heterogeneous systems [S2].
     
     It relates to making different query languages and data models work together [S4].
     
     It describes the ability of data (including documents, multimedia content and digital resources) to be universally accessible, reusable and comprehensible by all transaction parties (in a human-to-machine and machine-tomachine basis), by addressing the lack of common understanding caused by the use of different representations, different purposes, different contexts, and different syntax-dependent approaches [S6].
     
     It is defined as the ability of data (including documents, multimedia content, and digital resources) to be universally accessible, reusable, and comprehensible by all transaction parties (in a human-to-machine and machine-to-machine basis) by addressing the lack of common understanding caused by the use of different representations, different purposes, different contexts, and different syntax-dependent approaches [S14, S16].
     
     %It ensures that legislation does not impose unjustified barriers to data reuse in different policy areas [S15].
     
     It refers to make different data models and query languages working together [S19].
     
     It is related to data acquisition among several different devices and shared among application layers [S28]. 
     \\\hline 
     
     
    Device
    &
    
    It refers to enabling the integration and interoperability of such heterogeneous devices with various communication protocols and standards supported by heterogeneous IoT services [24]. 
    
    It provides information exchange between physical and software components of the smart devices including communication protocols; where the heterogeneity of application layer protocols is a primary concern [S28]. 
\\\hline
    Ecosystems & It is the ability of instant and seamless collaboration between different ecosystems and independent entities, entities within the ecosystems, and the ability of different independent entities to formulate virtual structures for specific purposes [S4], [S6].
    \\\hline
    
    Electronic Identity & It refers to the ability of different electronic identity systems within or across the boundaries of an enterprise to collaborate in order to automatically authenticate and authorize entities and to pass on security roles and permissions to the corresponding electronic identity holders, regardless of the system that they originate from [S4], [S6].
    \\\hline
    Enterprise 
    &
    It requires consideration of the enterprise from a general perspective, taking into account not only its different components and their interactions but also the environment in which it evolves and the interface through which it communicates with its environment [S2].
    
    It is concerned with interoperability between organizational units or business processes, either within a large distributed enterprise or within a network of enterprises [S14, S16].
% \\\hline    
%     Federated * \red{[S13]}
%     &
%     There is no common format, to establish interoperability, parties must accommodate, using a federated approach implies that no partner imposes its models, languages, and methods of work. This means that they must share an ontology to map their concepts at the semantic level \citep{chen2008architectures}.
\\\hline
    Functional
    &
    It is the capability to reliably exchange information without error [S14].
\\\hline 
    Hardware
    &
    It involves the integration of different computers, computer networks, etc. At this level network protocols are used so that two or more networks can communicate [S23].
\\\hline 
    Information
    &
    It is the ability of processes and systems to effectively exchange and use information services [S14, S16].
% \\\hline
%     Integrated * \red{[S13]
%     &
%     There exists a detailed common format for all models. The common format is not necessarily a standard but must be agreed upon by all parties to elaborate models and build systems \citep{chen2008architectures}.} 
\\\hline 
    Knowledge & It is the ability of two or more different entities to share their intellectual assets, take immediate advantage of the mutual knowledge and utilize it, and to further extend them through cooperation [S4], [S6].
\\\hline
    Legal
    &
     It is about ensuring that organizations operating under different legal frameworks, policies and strategies are able to work together [S5].
     
    It encompasses legislation issues involving the alignment of higher enterprise functions or government policies, usually to be expressed in the form of legal elements and business rules [S21].
    \\\hline 
    Network
    &
    It concerns with required to deal with seamless communication of devices over different networks [S28].
    \\\hline 
    
     Objects & 
    It refers to the networked interconnection and cooperation of everyday objects. These objects can embrace aspects besides and beyond software components, consistent with the concept of the Internet of Things [S4], [S6]. 
\\\hline
      Operational
    &
        
    It is the relation between/among actors cooperating to achieve a common goal, an overall, mutual capability necessary to ensure successful and efficient cooperation [S14].
    
    It is related to the process indicators related to cost, time, and process failure reduction [S21]. 
\\\hline
    Organizational
    &
    It concerns the business unit, process and people interactions across organization borders [S1]
      
    It facilitates the integration of business processes and workflows beyond the boundaries of a single organization [S3].
    
    It pertains to the capability of organizations to effectively communicate and transfer meaningful data (information) despite the use of a variety of information systems over significantly different types of infrastructure, possibly across various geographic regions and cultures [S4], [S6].
    
    It refers to the way in which public administrations align their business processes, responsibilities, and expectations to achieve commonly agreed and mutually beneficial goals [S5].
    
    It requests formal agreements on the conditions applicable to cross-organizational interactions [S15].
    
    It is concerned with business rules, policies and constraints, process alignment, and the actions necessary to make the entities collaborate [S17].
    
    It creates cohesion amongst approaches to governance, finance, legislation, and business processes [S19].
    
    It is concerned with defining business goals, modeling business processes and collaboration of administrations that wish to exchange information and may have different internal structures and processes [S20].
    
    %It refers to how participants’ systems align their processes, responsibilities, and expectations to achieve commonly agreed goals \red{[S15]}. 
    
    It involves the identification of the inter-actors and organizational procedures [S25].  
    
    It includes legal, political, or even cultural aspects of the institutions that participate in data sharing [S27].
    \\\hline
    Platform
    &
    It concerns the offers collaboration of the diverse platforms used in IoT due to diverse operating systems, programming languages, and access [S23].
    
    It enables interoperability across separate IoT platforms specific to one vertical domain such as smart home, smart healthcare, smart garden, etc. [S24].
    
    It offers a collaboration of the diverse platforms used in IoT due to diverse operating systems, programming languages, and access mechanisms for data and things [S28].
    \\\hline
    Pragmatic
    & 
    It is when the sender and the receiver of the message share the same expectations about the effect of the messages exchanged, and the context in which this exchange takes place plays an important role [S10].
    
    It encompasses the activities related to the management of one program in the context of another program [S21]
\\\hline
    Process
    &
    It makes various processes work together. In the networked enterprise, the aim will be to connect the internal processes of two companies to create a common process [S2, S19].
    
    It intends to make various processes work together. A process refers to the sequence of functions or services depending on company needs [S4].
    
    It is defined as the ability to align processes of different entities (enterprises), in order for them to exchange data and to conduct business in a seamless way [S6].
    
    It is the ability of diverse business processes to work together, to interoperate [S14].
     \\\hline
 
    Programmatic
    &
    It is concerned with ensuring that the message sender and receiver share the same expectations about the effect of the exchanged messages and the context where this exchange occurs plays an important role [S10].
    
    It is the ability of a set of communicating entities engaged in acquisition management activities to exchange specified acquisition management information and operate on that acquisition management information according to specified, agreed-upon operational semantic [S14].
\\\hline
    Rules & The ability of entities to align and match their business and legal rules for conducting legitimate automated transactions that are also compatible with the internal business operation rules of each other [S4], [S6].
\\\hline
    Semantic
    &
    
    It ensures the sharing of information and service for preserving the semantic flow [S1].
    
    It enables multiple systems to interpret the information that has been exchanged in a similar way through pre-defined shared meaning of concepts [S3]. 
   
    It is defined as the ability to operate on that data according to agreed-upon semantics [S4]. 
    
    It is pursued by the meaning of data elements and the relationships between them [S5].
    
    It is normally related to the definition of content, and deals with the human rather than machine interpretation of this content [S6].
    
    It expresses and understands the same information [S9, S11].
    
    It is concerned with ensuring that the meaning of the data, in other words, which the data refers to, is shared unambiguously way [S10].
    
    It is achievable when the captured information and knowledge can be effectively exchanged in a collaborative environment without any information and knowledge meaning and intent loss during this process [S13]. 
    
    It ensures the use of common descriptions of exchanged data [S15].
    
    It is the ability of systems to exchange information with unambiguous meaning [S16].
    
    It concerns the interpretation and mutual understanding between the interacting entities [S17]. 
    
    It is when systems exchange information with unequivocal meaning, ensuring that data meaning is shared unequivocally [S18]. 
    
    It enables collaborating systems to exchange and use the information using the correct meaning and provides the means and tools for automatic integration and processing of information without the intervention of humans [S20].
    
    It is refers to the ability of two or more computational systems to exchange information through a shared meaning that can be interpreted automatically and correctly [S22].
    
    It encompasses the intended meaning of the concepts in the data schema [S23].
    
    It is the ability to communicate entities to infer the correct "meaning" of messages [S24].
     
    It enables a seamless integration of different data sources and leverages risk identification. Related to the business-level understanding between different actors [S25].
    
    It is related to the common understanding of the meaning of certain data; a vocabulary (i.e., ontology) of the terms used in that specific context has to be shared first [26]. 
    
    It is about making sure that the shared information has the same meanings between different institutions or countries [27].
    
    It is linked with the meaning of the content for humans rather than machine interpretation of the content [S28]. 
    
    It is the ability, of health information systems, to exchange information and automatically interpret the information exchanged meaningfully and accurately in order to produce useful results as defined by the end users of both systems [S29].
    
    It aims to share data among organizations or systems and ensure they understand and interpret data regardless of who is involved, using domain concepts, context knowledge, and formal data representation [S30]. 
    \\\hline 
    
    Service
    &
    
    It makes it possible for various services or applications (designed and implemented independently) to work together by solving the syntactic and semantic differences [S2].
    
    It refers to identifying, composing, and making various applications that are implemented and designed independently function together [S4. S16].
    
    It is a concern of a company to dynamically register, aggregate and consume services composed from an external source. It corresponds to resource sharing in the design of new cloud-based data services as external sources. Also, this type of interoperability present the exchange of information between geographically distributed multidisciplinary teams [S19].
\\\hline

    Social Networks & It refers to the ability of enterprises to seamlessly interconnect and utilize social networks for collaboration purposes, by aligning their internal structure to the fundamental aspects of the social networks [S4], [S6].
    \\\hline
    
    Software Systems & 
    It refers to the ability of an enterprise system or a product to work with other enterprise systems or products without special effort from the stakeholders [S4], [S6].
    \\\hline
    Syntactic & 
    
    It guarantees the preservation of the clinical purpose of the data during transmission among healthcare systems [S3].
    
    It is defined as the ability to exchange data. Syntactic interoperability is generally associated with data formats. The messages transferred by communication protocols should possess a well-defined syntax and encoding, even if only in the form of bit-tables [S4], [S6].
    
    It is related to the data that are exchanged act as a sign and, to achieve this interoperability level, the sign syntax must be previously established as a standard [S10]. 
    
    It is concerned with communication, data exchange, and syntax consistency [S17]. 
    
    It concerns the information format to be exchanged [S21]
    
    It refers to interoperation of the format as well as the data structure used in any exchanged information or service between heterogeneous IoT system entities [S24].
    
    It deals with the format of messages exchanged between systems [26]. 
    
    It should include a data validation process related to the format, syntax, grammar, or schema [S27].
    \\\hline
    System 
    &
    It is the ability of systems to operate together, with systems defined in line with the generic combination of interacting elements organized to achieve one or more stated purposes [S14, S16].
\\\hline
    Technical & 
    
    It ensures the continuity of the  semantic flow (e.g. technology solutions, standards and tools for the exchange of data between IS) [S1].
    
    It enables heterogeneous systems to exchange data, but it does not guarantee that the receiving system with be able to use the exchanged data in a meaningful way [S3].
    
    It is achieved among communications electronics systems or items of communications electronics equipment when services or information could be exchanged directly and satisfactorily between them and their users [S4].
    
    It covers the applications and infrastructures linking systems and services. Aspects of technical interoperability include interface specifications, interconnection services, data integration services, data presentation and exchange, and secure communication protocols [S5].
    
    It is achieved among communication selectronics systems or items of communications-electronics equipment when services or information could be exchanged directly and satisfactorily between them and their users [S6].
    
    It is the ability of systems to provide dynamic interactive information and data exchange among systems [S12].
    
    It is the ability achieved by communication and electronic systems when information or services can be exchanged directly and satisfactorily between them and/or their use [S14, 16].

    It is related to setting up the necessary information systems environment to allow an uninterrupted flow of bits and bytes [S15].
    
    It concerns with the connectivity, communication, and operation regarding the interacting entities, and middleware elements regarding authentication and authorization, the use of technical standards, protocols for communication and transport, and interfaces between components [S17]. 
    
    It is concerned with the technical issues of linking up computer systems for sharing information [S20]

    It covers the applications and infrastructures linking systems and services. It includes interface specifications, interconnection and data integration services, data presentation and exchanging, and secure communication protocols [S21].

    
    It is related to the standardization of hardware and software interfaces [S25]. 
    
    In the health context, this interoperability is achieved by directing exchanged information to the smart e-Health gateway, which has multiple interfaces [26].
    
    It ensures information exchange requirements between different systems [S27].
    \\\hline
%     Unified * \red{[S13]}
%     &
%     There exists a common format but only at a meta-level. This meta-model is not an executable entity as it is in the integrated approach but provides a means for semantic equivalence to allow mapping between models \citep{chen2008architectures}.
% \\\hline

\end{longtable}
%\twocolumn
\normalsize
%\end{landscape}
